Wireless sensor networks (WSNs), composed of a large number of low-cost, battery-powered sensors, and a relatively powerful sink node, have recently emerged as promising computing platforms for many non-traditional applications. 

Although the code running on the sensors is preloaded to them before deployment, it may still need updates for many reasons. 
In single application wireless sensor networks (SA-WSNs), sensors need to upgrade the software in order for the WSN to adapt to the changing demands of the users.
In multiple application wireless sensor networks (MA-WSNs), each sensor may have to switch between different applications upon request. Due to the memory size limitation, the sensors may not be able to store the complete image of all the applications in the local memory. 
Thus, the sensors may have to convert a temporarily unwanted application to the wanted application by applying software updates.

Despite the fact that the sensors need such code updates for various reasons, the code update patches are often transmitted via wireless channels, because the sensors are usually left unattended after deployment. As the code is transmitted via battery-powered wireless communication, the energy consumed in the software update can be significant, especially when it happens frequently.

The goal of this research is to design a software update management framework, which
optimizes the energy consumption in WSN software updates. The proposed framework
includes an update-conscious compiler, a patch script generator, and a code dissemination
protocol. First, it generates the new binary image using the update-conscious compilation
techniques, and then compares the new binary with the old binary to generate the patch.
The update-conscious techniques make the new binary code more similar to the old version,
so that the size of the patch script can be reduced. The patch generator summarizes high
level binary differences in a script. This technique furthermore reduces the script size.
Finally, the framework transmits the generated patch over the network using an efficient code
dissemination protocol. This technique reduces the time and energy spent in propagating
the update patches over the network. After the sensor nodes receive the complete patch,
they will regenerate the target executable.

This research solves an important problem in WSN study. The designed software update framework will benefit all the WSN users by making the software update procedure faster and more energy efficient. Besides that, it is also the first research into update-conscious compilation techniques. Proposing the problem of generating similar binaries from source code that has just a few changes is another contribution of this research. 
