Wireless sensor networks (WSNs), composed of a large number of low-cost, battery-powered sensors, and a relatively 
powerful sink node, have recently emerged as promising computing platforms for many non-traditional applications. 
The code running on the sensors is preloaded to them before deployment, but it still needs updates for 
bug fix or feature enhancement. The code update patches are often 
transmitted via wireless channels, because the sensors are usually left unattended after deployment. 

As the patch is 
transmitted via wireless communication, the energy consumed in the software update can be significant, 
especially when it happens frequently.
Because the sensors are powered with batterers thus running with limited energy supplies,
consuming too much energy in the software update procedure can shorten the life time of the sensors as well as the
WSN. Therefore, designing an efficient software update method for WSNs is very critical.

This dissertation proposes an efficient WSN software update framework that contains novel solutions to this problem.
A WSN software update procedure can be divided into two phases: the patch preparation and patch distribution.
In the former phase, the new binary image is first generated by a compiler, and then the binary
level {\it diff} between the old and new binary is used to form the update patch. 
In the latter phase, the patch packets are disseminated from source to destinations in the network. 
In the patch preparation phase, the proposed framework diminishes energy consumption of WSN software update by 
decreasing the patch size.
It first uses an update-conscious compiler (UCC) that strives to match the old
compilation decisions when generating the new binary. This technology improves the binary level similarity, thus, reduces
the {\it diff} that needs to be incorporated in the patch.
It then uses a well designed patch formating method that can present multiple 
binary level differences in one primitive, which makes the patch even smaller.
The experimental results show that these two methods workings together can 
reduce the patch size by 81\% for general purpose applications and 61\% for DSP applications.
This dissertation proposes a multicast-based code redistribution protocol (MCP) to 
effectively disseminate the minimized patch to individual sensors.
By saving the routing information on-site and using caching recently received packets, 
MCP can achieve a faster code delivery with less network traffic.
The experimental results show that MCP reduces the network traffic by 56\% and the code distribution time by 73\%.

In summary, this dissertation solves an important problem in WSN study, and it is also the first 
research into update-conscious compilation techniques. 
The designed software update framework will benefit all the WSN 
users by making the software update procedure faster and more energy efficient. 
In the future work, more update-conscious compilation techniques will be approached. 
Also, the proposed techniques can be adapted to other hardware platforms, e.g., smartphones.
Similar energy savings are expected to be seen, while applying them to other platforms.

