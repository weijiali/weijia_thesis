Wireless sensor networks (WSNs), composed of a large number of low-cost, battery-powered sensors, and a relatively powerful sink node, have recently emerged as promising computing platforms for many non-traditional applications. 

Although the code running on the sensors is preloaded to them before deployment, it may still need updates for many reasons. 
In single application wireless sensor networks (SA-WSNs), sensors need to upgrade the software in order for the WSNs to adapt to the changing demands of the users.
In multiple application wireless sensor networks (MA-WSNs), each sensor has to switch between different applications upon request. Due to the memory size limitation, the sensors might not be able to store the complete application images in the local memory. Thus, the sensors may have to apply certain updates to a temporarily unwanted application, and convert it into the wanted application.

Despite the fact that the sensors need such code updates for various reasons, the code update patches are often transmitted via wireless channels, because the sensors are usually left unattended after deployment. As the code is transmitted via battery-powered wireless communication, the energy consumed in the software update can be significant, especially when it happens frequently.

The goal of this research is to design a software update management framework, which optimizes the overall energy consumption in a WSN software update. The proposed framework includes an update-conscious compiler, a patch script generator, and a code dissemination protocol.
First, it generates the new binary image using the update-conscious compilation technique. This technique makes newly generated binary code more similar as the old version, in order to reduce the size of the patch script which is generated by simply comparing two binary images. Then, it generates the update patch using a high compression ratio patch generator. This technique furthermore reduces the script size. 
Finally, it transmits the generated patch over the network using an efficient code dissemination protocol. This technique reduces the time and energy spent in propagating the update patches over the network. After the sensor nodes receive the complete patches, they will run a light weight code retriever to regenerate the executable binary.

%Both energy consumption and time consumption is measured to evaluate the performance of the framework design. The overall energy consumption includes not only the code transmission overhead in the network, but also the patch generation overhead on the sensor nodes and the long time running overhead of the updated software. The time consumption includes the code distribution time consumption and code rebuild time consumption.

This research solves an important problem in WSN study. The proposed software update framework will benefit all the WSNs users by making the software update procedure faster and more energy efficient. Besides that, it is also the very first research of update-conscious compilation techniques. The motivation of similar code generation in compilation research is another great contribution of this research. 