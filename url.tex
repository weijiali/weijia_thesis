\documentclass[a4paper]{article}
\usepackage[T1]{fontenc}
\usepackage{lmodern}
\usepackage{miscdoc,url,booktabs}
\begin{document}
\title{\texttt{url.sty} version 3.3}
\date{2006-04-12, documentation 2010-01-20}
\author{Donald Arseneau\\
  edited as a \LaTeX{} document by Robin Fairbairns}
\maketitle

% url.sty  ver 3.3    12-Apr-2006   Donald Arseneau   asnd@triumf.ca
% Copyright 1996-2006 Donald Arseneau,  Vancouver, Canada.
% This program can be used, distributed, and modified under the terms
% of the LaTeX Project Public License.

The package defines a form of \cs{verb} command that allows linebreaks
at certain characters or combinations of characters, accepts
reconfiguration, and can usually be used in the argument to another
command.  It is intended for email addresses, hypertext links,
directories/paths, etc., which normally have no spaces.  The font used
may be selected using the \cs{urlstyle} command, and new url-like
commands may be defined using \cs{urldef}.

\begin{center}
  \begin{tabular}{lp{3.5in}}
    \toprule 
    Usage &    Conditions \\
    \midrule
    \verb+\url{ }+ & If the argument contains any ``\verb+%+'',
                       ``\verb+#+'' or ``\verb+^^+'', or ends with
                       ``\cs{}'', it can't be used in the argument to
                       another command.\\
                     & The argument must not contain unbalanced braces.\\
    \verb+\url|  |+ & where ``\texttt{|}'' is any character not used
                        in the argument and not ``\verb+{+'' or a
                        space.  The same restrictions apply as above
                        except that the argument may contain
                        unbalanced braces.\\ 
    \cs{xyz} & for ``\cs{xyz}'' a defined-url; such a command can be
               used anywhere, no matter what characters it contains. \\
      \bottomrule
    \end{tabular}
  \end{center}

The ``\cs{url}'' command is fragile, and its argument is likely to be
very fragile, but a defined-url is robust.

\section{Package options}

\newitem Package Option:  \texttt{obeyspaces}

Ordinarily, all spaces are ignored in the url-text.  The
``\texttt{[obeyspaces]}'' option allows spaces, but may introduce
spurious spaces when a url containing ``\cs{}'' characters is given in
the argument to another command. 

So if you need to obey spaces you can say
``\cs{usepackage}\texttt{[obeyspaces]\char`\{url\char`\}}'', and if
you need both spaces and backslashes, use a defined-url.

\newitem Package Option:  \texttt{hyphens}

Ordinarily, breaks are not allowed after ``\texttt{-}'' characters
because this leads to confusion.  (Is the ``\texttt{-}'' part of the
address or just a hyphen?)

The package option ``\texttt{[hyphens]}'' allows breaks after explicit
hyphen characters.  The \cs{url} command will \textbf{never ever}
hyphenate words.

\newitem Package Option:  \texttt{spaces}

Likewise, breaks are not usually allowed after spaces under the
``\texttt{[obeyspaces]}'' option, but if you give the options
``\texttt{[obeyspaces,spaces]}'', \cs{url} will allow breaks at those
spaces.
\begin{quote}
  Note that it seems logical to allow the sole option
  ``\texttt{[spaces]}'' to let input spaces indicate break points, but
  not to display them in the output.  This would be easy to implement,
  but is left out to avoid(?) confusion.
\end{quote}

\newitem Package Option:  \texttt{lowtilde}

Normal treatment of the \verb+~+ character is to use the font's
``\cs{textasciitilde}'' character, if it has one (or claims to).
Otherwise, the character is faked using a mathematical ``\cs{sim}''.
The ``\texttt{[lowtilde]}'' option causes a faked character to be used
always (and a bit lower than usual).

\section{Defining a defined-url}

Take for example the email address ``\url{myself%node@gateway.net}"
which could not be given (using ``\cs{url}'' or ``\cs{verb}'') in a
caption or parbox due to the percent sign.  This address can be
predefined with
\begin{quote}
  \verb|\urldef{\myself}\url{myself%node@gateway.net}| or\\
  \verb+\urldef{\myself}\url|myself%node@gateway.net|+
\end{quote}
and then you may use ``\cs{myself}'' instead of
``\verb+\url{myself%node@gateway.net}+''
in an argument, and even in a moving argument like a caption because a
defined-url is robust.

\section{Style}

You can switch the style of printing using ``\verb+\urlstyle{xx}+'',
where ``\emph{\texttt{xx}}'' can be any defined style.  The
pre-defined styles are ``\texttt{tt}'', ``\texttt{rm}'',
``\texttt{sf}'' and ``\texttt{same}'' which all allow the same
linebreaks but use different fonts~--- the first three select a
specific font and the ``\texttt{same}'' style uses the current text
font.  You can define your own styles with different fonts and/or
line-breaking by following the explanations below.  The ``\cs{url}''
command follows whatever the currently-set style dictates.

\section{Alternate commands}

It may be desireable to have different things treated differently, each
in a predefined style; e.g., if you want directory paths to always be
in typewriter and email addresses to be roman, then you would define new
url-like commands as follows:
\begin{quote}
\begin{verbatim}
\DeclareUrlCommand\email{\urlstyle{rm}}
\DeclareUrlCommand\directory{\urlstyle{tt}}
\end{verbatim}
\end{quote}

In fact, this \cs{directory} example is exactly the \cs{path}
definition which is pre-defined in the package.  If you look above,
you will see that \cs{url} is defined with
\begin{quote}
\begin{verbatim}
\DeclareUrlCommand\url{}
\end{verbatim}
\end{quote}
I.e., using whatever \cs{urlstyle} and other settings are already in
effect.

You can make a defined-url for these other styles, using the usual
\cs{urldef} command as in this example:
\begin{quote}
\begin{verbatim}
\urldef{\myself}{\email}{myself%node.domain@gateway.net}
\end{verbatim}
\end{quote}
which makes \cs{myself} act like
\verb+\email{myself%node.domain@gateway.net}+,
if the \cs{email} command is defined as above.  The \cs{myself}
command would then be robust.

\section{Defining styles}

Before describing how to customize the printing style, it is best to
mention something about the unusual implementation of \cs{url}.  Although
the material is textual in nature, and the font specification required
is a text-font command, the text is actually typeset in \emph{math} mode.
This allows the context-sensitive linebreaking, but also accounts for
the default behavior of ignoring spaces.  Now on to defining styles.

To change the font or the list of characters that allow linebreaks, you
could redefine the commands \cs{UrlFont}, \cs{UrlBreaks},
\cs{UrlSpecials}, etc., directly in the document, but it is better to
define a new `url-style' (following the example of \cs{url@ttstyle}
and \cs{url@rmstyle}) which defines all of \cs{UrlBigbreaks},
\cs{UrlNoBreaks}, \cs{UrlBreaks}, \cs{UrlSpecials}, and \cs{UrlFont}.

\subsection{Changing font}
The \cs{UrlFont} command selects the font.  The definition of
\cs{UrlFont} done by the pre-defined styles varies to cope with a
variety of \LaTeX{} font selection schemes, but it could be as simple
as \verb+\def\UrlFont{\tt}+.  Depending on the font selected, some
characters may need to be defined in the \cs{UrlSpecials} list because
many fonts don't contain all the standard input characters.

\subsection{Changing linebreaks}
The list of characters that allow line-breaks is given by
\cs{UrlBreaks} and \hspace{1in}\penalty-1\hspace{-1in}
\cs{UrlBigBreaks}, which have the format
\cs{do}\cs{c} for each character \texttt{c}.

The differences are that `BigBreaks' usually have a lower penalty and
have different breakpoints when in sequence (as in \texttt{http://}):
`BigBreaks' are treated as mathrels while `Breaks' are mathbins (see
The TeXbook, p.170). In particular, a series of `BigBreak' characters
will break at the end and only at the end; a series of `Break'
characters will break after the first and after every following
\emph{pair}; there will be no break after a `Break' character if a
`BigBreak' follows.  In the case of \texttt{http://} it doesn't matter
whether \texttt{:"} is a `Break' or `BigBreak'~--- the breaks are the
same in either case; but for \emph{DECnet} nodes with \texttt{::} it
is important to prevent breaks \emph{between} the colons, and that is
why colons are `BigBreaks'.

It is possible for characters to prevent breaks after the next
following character (I use this for parentheses).  Specify these in
\cs{UrlNoBreaks}.

You can do arbitrarily complex things with characters by making them
active in math mode (mathcode hex-8000) and specifying the
definition(s) in \cs{UrlSpecials}.  This is used in the \texttt{rm}
and \texttt{sf} styles for OT1 font encoding to handle several
characters that are not present in those computer-modern style fonts.
See the definition of \cs{Url@do}, which is used by both
\cs{urlrmstyle} and \cs{url@sfstyle}; it handles missing characters
via \cs{UrlSpecials}.  The nominal format for setting each special
character \texttt{c} is: \verb+\do\c{+\meta{definition}\verb+}+, but
you can include other definitions too.

If all this sounds confusing~\dots{} well, it is!  But I hope you
won't need to redefine breakpoints~--- the default assignments seem to
work well for a wide variety of applications.  If you do need to make
changes, you can test for breakpoints using regular math mode and the
characters \texttt{+=(a}. 

You can allow some spacing around the breakable characters by assigning
\begin{quote}
\begin{verbatim}
\Urlmuskip = 0mu plus 1mu
\end{verbatim}
\end{quote}
You can change the penalties used for BigBreaks and Breaks by assigning
\begin{quote}
\begin{verbatim}
\mathchardef\UrlBreakPenalty=100
\mathchardef\UrlBigBreakPenalty=100
\end{verbatim}
\end{quote}
The default penalties are \cs{binoppenalty} and \cs{relpenalty}.
These have such odd non-\LaTeX{} syntax because I don't expect people
to need to change them often.

\section{Yet more flexibility}
You can also customize the verbatim text by defining \cs{UrlRight}
and/or \cs{UrlLeft}, e.g., for ISO formatting of urls surrounded by
\verb+<  >+, define
\begin{quote}
\begin{verbatim}
\DeclareUrlCommand\url{\def\UrlLeft{<url:\ }\def\UrlRight{>}%
    \urlstyle{tt}}
\end{verbatim}
\end{quote}
The meanings of \cs{UrlLeft} and \cs{UrlRight} are \emph{not}
reproduced verbatim.  This lets you use formatting commands there, but
you must be careful not to use TeX's special characters
(\verb+\^_%~#$&{}+ etc.) improperly.  You can also define \cs{UrlLeft}
to reprocess the verbatim text, but the format of the definition is special:
\begin{quote}
\begin{verbatim}
\def\UrlLeft#1\UrlRight{ ... do things with #1 ... }
\end{verbatim}
\end{quote}
Yes, that is \verb+#1+ followed by \cs{UrlRight} then the definition.  For
example, to put a hyperTeX hypertext link in the DVI file:
\begin{quote}
\begin{verbatim}
\def\UrlLeft#1\UrlRight{\special{html:<a href="#1">}#1\special{html:</a>}}
\end{verbatim}
\end{quote}
Using this technique, \path{url.sty} can provide a convenient
interface for performing various operations on verbatim text.  You
don't even need to print out the argument!  For greatest efficiency in
such obscure applications, you can define a null url-style where all
the lists like \cs{UrlBreaks} are empty.


\noindent\begin{verbatim}
%
% Revision History:
% ver 1.1 6-Feb-1996:
% Fix hyphens that wouldn't break and ligatures that weren't suppressed.
% ver 1.2 19-Oct-1996:
% Package option for T1 encoding; Hooks: "\UrlLeft" and "\UrlRight".
% ver 1.3 21-Jul-1997:
% Prohibit spaces as delimiter characters; change ascii tilde in OT1.
% ver 1.4 02-Mar-1999:
% LaTeX license; moving-argument-error
% ver 1.5 28-Mar-1999:
% possibility of spacing around break characters; re-settable penalties
% ver 1.6 20-Jun-2002:
% un-double #, fix obeyed-spaces, ignore trailing %, hook for hyperref
% (\Url@HyperHook), no macros in pre-processed url string (in \Url@String),
% limit catcode change of ~.
% ver 3.0 June 2003/Nov 2003:
% \DeclareUrlCommand; make font encoding automatic (only a few inputenc characters
% are supported yet - needs refactoring); reverse penalties. 
% ver 3.1 Mar 2004:
% Remove spurious spaces in \url@XXstyle commands.
% ver 3.2 June 2005:
% Fix cmsy-symbols in tt bug (from 3.0); LY1 encoding bug; Enable plain
% with miniltx (again); Define the \urldef for hyperref; Lower "sim" tilde
% a little; fix \lowercase error in \UrlSpecials handling.
% ver 3.3 April 2006:
% Fix some encoding bugs and remove 8-bit characters.  lowtilde option
% The End
\end{verbatim}
\end{document}

Test file integrity:  ASCII 32-57, 58-126:  !"#$%&'()*+,-./0123456789
:;<=>?@ABCDEFGHIJKLMNOPQRSTUVWXYZ[\]^_`abcdefghijklmnopqrstuvwxyz{|}~


%%% Local Variables: 
%%% mode: latex
%%% TeX-master: t
%%% End: 
